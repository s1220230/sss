%tex file for ISCIE International Symposium on 
%            Stochastic Systems Theory and Its Applications
%
%

%latex209
%\documentstyle[sss,epsfig]{article}

%latex2e
\documentclass[a4paper]{article}
\usepackage{latexsym}
\usepackage{sss}
\usepackage{graphicsx} % for pdf, bitmapped graphics files
\usepackage{epsfig} % for postscript graphics files


\begin{document}
\date{}
\title{\LARGE{\bf
The 50th ISCIE International Symposium on Stochastic\\
Systems Theory and Its Applications\\
Instruction for Authors}
}
\author{
Hirokazu Ohtagaki \\
Dept. of Electrical and Electronic Systems, Okayama University of Science\\
Ridai-cho 1-1, Kita-ku, Okayama, 700-0005 Japan\\
\\E-mail: \texttt{xxx@sci-sss.org}
}

\maketitle
\thispagestyle{empty}
%ABSTRACT
\abstract{
This document provides instructions for authors for preparing and submitting
the manuscripts for the Proceedings of the 49th ISCIE International Symposium
on Stochastic System Theory and Its Applications.
This document (including this abstract) itself is an example of the desired layout
for camera-ready papers.
The guidelines detailed in this document will enable the Conference to maintain
uniformity in the final proceedings and to avoid additional publication charges.
To ensure high-quality proceedings, readability of your original manuscript is of
paramount importance.
Also, adherence to deadlines is critical because a great portion of the publication
process is performed sequentially.  
}


\section{Introduction}

The manuscript for the Proceedings should be prepared in accordance with the present
instructions.
The Symposium Proceedings will be published at the on-line electronic journal cite 
J-Stage (Japan Science and Technology Information Aggregator, Electronic). 
Authors are requested to prepare the camera-ready manuscripts on {\bf A4} paper,
written in English, in {\bf PDF} format.
The size of the electronic file should not exceed {\bf 5MB}.

The manuscript should not exceed {\bf 10 pages} (6 pages are recommended) including
figures, tables, photos, references and appendices.
The author who wishes to write the manuscript more than 10 pages or more than 5MB
memory should contact with the SSS'17 Steering Committee for permission.
 
On A4 paper, the Conference requires a font size of {\bf 10 points or greater}.
Some technical formatting programs print mathematical formulas in italic type, with
subscripts and superscripts in a slightly smaller font size. 

In order to assure a high-quality Proceedings, all manuscripts will be reviewed and
only accepted ones will appear in the Proceedings.


\section{Preparation of Exact Size Manuscripts}

Your paper must be prepared in actual size (i.e., exactly how it is to appear in the
proceedings) with {\bf single spacing in two columns}.
To ensure uniformity of appearance for the proceedings, the papers should conform to
the following specifications.
\begin{enumerate}
\item[$\bullet$]
The top margin of the first page, i.e. the distance from the top edge of the paper
to the title, should be {\bf 30mm}.
\item[$\bullet$]
The top margins of the second, and the subsequent pages should be {\bf 26mm}.
\item[$\bullet$]
The text should be centered, and both the left and right margins should be {\bf 16mm}.
\item[$\bullet$]
The column width should be {\bf 84mm}.
\item[$\bullet$]
The space between the two columns is {\bf 10mm}.
\item[$\bullet$]
The bottom margins for all pages should be no less than {\bf 26mm}.
\end{enumerate}

If your paper deviates significantly from these specifications, it may not be
included in the proceedings. 


\section{Manuscript Style}

Regarding the styles of your manuscript, please conform to the following 
instructions. 

\subsection{Title}
The title should be centered across the top of the first page and should 
be in a distinctive point size or font.

\subsection{Author's Names and Address}
The authors' names and addresses should be centered below the title.
It is desirable that these lines are typed in at least eleven point font size,
but the particular point sizes and fonts are not critical and are left to the
direction of the authors.
Times new Roman 12 point is suggested.
Please include your E-Mail address. 

\subsection{Headings}
Main headings are to be column centered in a bold font without an underline.
They may be numbered, if so desired. 

Subheadings should be in a bold font or underlined lowercase with initial
capitals.
They should start at the left-hand margin on a separate line.

Sub-subheadings are to be in a bold font or underlined type.
They should be indented and run in at the beginning of the paragraph.

\subsection{Figures and Tables} 
Figures and photos should be consecutively numbered like Fig{.}~1, Fig{.}~2,
Figures should be inserted near their citation or at the end of the manuscript.
Large figures and tables may span across both columns if necessary.
Figure captions should be placed below the figures. 

\subsection{References}
List and number all references at the end of the paper as shown below.
Number reference citations consecutively in square brackets [1].

\subsection{Page Numbers}
Do not write page numbers on your manuscript.
These will be inserted later by the proceedings printer together with
the session number and conference identifications.


\section{Manuscript Submission}
Authors are requested to send their manuscripts electronically by
{\bf July 31, 2018} on the web:
\begin{center}
http://sci-sss.org/sss2018/sub/submission.php
\end{center}


\section{Conclusions}

Please make an extra effort to adhere to these guidelines as the quality
of the publications depends on you.
Thank you for your cooperation and contribution.
We are looking forward to seeing you at the 49th ISCIE International Symposium
on Stochastic Systems Theory and Its Applications (SSS'17).


\begin{thebibliography}{99}
\bibitem{foo}
R. E. Kalman and R. S. Bucy: 
New Results in linear filtering and prediction, 
{\it Trans. ASME, J. Basic Eng.}, 82 D, pp.95-108, 1960.
\bibitem{bar}
A. H. Jazwinsky: 
{\it Stochastic Process and Filtering Theory}, Academic Press, N.Y., 1970.
\end{thebibliography}
\end{document}

% end of sss10.tex
